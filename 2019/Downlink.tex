
\section{Downlink Telemetry Specifications}
\label{sec:Downlink}

\hspace*{0.5cm}
\begin{minipage}{\linewidth-0.5cm}
  \begin{enumerate}[label=\Alph*.]    
  \item Serial data downlink format: \newline
    \textbf{Packetized}
  \item Approximate serial downlink rate (in bits per second) \newline
    \SI{500}{\bit\per\second}.
  \item Specify your serial data record including record length and information contained in each record byte. \newline
    This can be seen in Table \ref{tab:SerialRecord}.
  \item Number of analog channels being used:
    No analog channels will be used.
  \item If analog channels are being used, what are they being used for?
  \item Number of discrete lines used: \newline   
    4 lines. Pins F, N, H, and P.
  \item If discrete lines are being used what are they being used for? \newline
    F - activate astrobiology system, N - deactivate astrobiology system, H - Reboot SOCRATES.
  \item Are there any on-board transmitters? If so, list the frequencies being used and the transmitted power. \newline
    No on-board transmitters.
  \item Other relevant downlink telemetry information? \newline
    The following is a sample data packet:
    
    1,44.3,51.1,2.12,3.11,53.6,5.22,2.32,0.02,1.01,27.3,9.21,5.33,4.65,7.35,
    
    7.24,5.73,9.35,3.25,7.31,5.42,4.21,7.25,221.1,375.5,311.2,11:03:43,06/05/19
    
    Each packet will be continuous with no breaks in the middle. Each packet will be broken by a newline character.
  \end{enumerate}
\end{minipage}

\begin{table}[h] 
  \caption{Serial record for the SORA 3 payload.}
  \label{tab:SerialRecord}
  \begin{tabularx}{0.8\linewidth}{Y Y}
    \hline
    \hline
    \multicolumn{1}{c}{Description} & \multicolumn{1}{c}{Byte Number} \\
    \hline
    packet\_num & 0 \\
    RPI Temp & 1 - 4 \\
    MiniPIX-0 Temp & 5 - 8 \\
    MiniPIX-0 Dose & 9 - 12 \\
    MiniPIX-0 Counts & 13 - 16 \\
    MiniPIX-1 Temp & 17 - 20 \\
    MiniPIX-1 Dose & 21 - 24 \\
    MiniPIX-1 Counts & 25 - 28 \\
    Ambient Pressure & 29 - 32 \\
    ISS Pressure & 33 - 36 \\
    ISS Temp & 37 - 40 \\
    Solar Cell Temps & 41 - 88 \\
    Photodiodes & 89 - 100 \\
    Timestamp & 101 - 116 \\
    end\_packet & 117 \\
    \hline
    \textbf{Total} & \textbf{118} \\
    \hline
    \hline
  \end{tabularx}
\end{table}
